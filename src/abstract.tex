%/**
% * LaTeX thesis template (abstract)
% * @author  : Alexander willner (willner@cs.uni-bonn.de)
% */
\chapter*{Abstract}\label{sec:abstract}
%\mtcaddchapter\addcontentsline{toc}{chapter}{Abstract}


\note{Introduction and Problem Statement}
Suspendisse vel felis. Ut lorem lorem, interdum eu, tincidunt sit amet, laoreet vitae, arcu. Aenean faucibus pede eu ante. Praesent enim elit, rutrum at, molestie non, nonummy vel, nisl. Ut lectus eros, malesuada sit amet, fermentum eu, sodales cursus, magna. Donec eu purus. Quisque vehicula, urna sed ultricies auctor, pede lorem egestas dui, et convallis elit erat sed nulla. Donec luctus. Curabitur et nunc. Aliquam dolor odio, commodo pretium, ultricies non, pharetra in, velit. Integer arcu est, nonummy in, fermentum faucibus, egestas vel, odio.

\note{Related Work and Research Questions}
\lipsum[6]

\note{Research Contributions and how they've been validated}
This thesis contribution is severalfold. At first...
\lipsum[6]

\note{Conclusions}
\lipsum[6]

%    \begin{quote}
%    "`\begin{CJK}{UTF8}{gbsn}不闻不若闻之,闻之不若见之,见之不若知之,知之不若行之;学至于行之而止矣。\end{CJK}"' -
%    \textit{Xunzi, Chinese philosopher (about 300-230 BC)}
%
%    "`not hearing is not as good as hearing, hearing is not as good as
%       seeing, seeing is not as good as mentally knowing, mentally knowing is
%       not as good as acting; true learning continues up to the point that
%       action comes forth [or, only when a thing produces action can it be said
%      to have been truly learned]"'
%    \end{quote}

\cleardoublepage
\chapter*{Zusammenfassung}\label{sec:zusammenfassung}
\note{Einleitung und Problemstellung}
\lipsum[6]

\note{Verwandte Arbeiten und Forschungsfragen}
\lipsum[6]

\note{Wissenschaftlicher Beitrag}
\lipsum[6]

\note{Zusammenfassung}
\lipsum[6]
\cleardoublepage
